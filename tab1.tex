\begin{table}[h!]
\caption{Tabela tarefas do Analista de sistemas}
\label{tab2}
\begin{tabular}{|l|l|l|}
\hline
\begin{tabular}[c]{@{}l@{}}Item / \\ papel\end{tabular} & \begin{tabular}[c]{@{}l@{}}Analista de \\ Sistemas\end{tabular}                 & Aplicações                                                                                                                                                                                                                                                                              \\\hline
1                                                       & \begin{tabular}[c]{@{}l@{}}Construir Modelo\\  de Dominio\end{tabular}          & \begin{tabular}[c]{@{}l@{}}Esta tarefa compreende a definição do\\  modelo de Dominio de alto nível.\end{tabular}                                                                                                                                                                       \\
2                                                       & \begin{tabular}[c]{@{}l@{}}Construir Modelo\\  de Negócio\end{tabular}          & \begin{tabular}[c]{@{}l@{}}Consiste em construir um \\ modelo do negocio de alto nivel.\end{tabular}                                                                                                                                                                                    \\\hline
3                                                       & \begin{tabular}[c]{@{}l@{}}Construir Casos de \\ Uso de Alto Nível\end{tabular} & \begin{tabular}[c]{@{}l@{}}Consiste em construir \\ casos de uso de alto nivel\end{tabular}                                                                                                                                                                                             \\\hline
4                                                       & \begin{tabular}[c]{@{}l@{}}Elaborar \\ casos de Teste\end{tabular}              & \begin{tabular}[c]{@{}l@{}}A elaboração dos casos de testes devem \\ objetivar a identificação dos erros \\ que eventualmente existam no sistema.\\  Além disso, tais documentos \\ servem como insumo para a verificação \\ se todos os requisitos estão sendo atendidos.\end{tabular} \\\hline
5                                                       & \begin{tabular}[c]{@{}l@{}}Prototipar a Interface \\ com o Usuário\end{tabular} & \begin{tabular}[c]{@{}l@{}}Desenvolver interfaces de fácil \\ intendimento para ajudar na coleta \\ de requisitos e no desenvolvimento do inclemento.\end{tabular}                                                                                                                      \\\hline
6                                                       & \begin{tabular}[c]{@{}l@{}}Treinar \\ Usuários\end{tabular}                     & \begin{tabular}[c]{@{}l@{}}Assim que liberada a versão deve apresentar ao usuário, \\ para que o mesmo faça a validação no treinamento.\end{tabular}                                                                                                                                    \\\hline
7                                                       & \begin{tabular}[c]{@{}l@{}}Revisão e \\ novos requisitos\end{tabular}           & \begin{tabular}[c]{@{}l@{}}Detalhar requisitos funcionais restantes. \\ Esse detalhamento é realizado em \\ conjunto com o Cliente e resulta na criação \\ de um documento de Requisitos\end{tabular}                                                                                   \\\hline
8                                                       & Implantação                                                                     & \begin{tabular}[c]{@{}l@{}}A implantação do sistema consiste da \\ entrega do produto final para o cliente. \\ Sendo necessário para isso a criação do \\ ambiente no qual o sistema será executado.\end{tabular} \\\hline

\end{tabular}
\end{table}
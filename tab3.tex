\begin{table}[h!]
\centering
\caption{Tabela tarefas do Desenvolvedor}
\label{tab3}
\begin{tabular}{|l|l|l|}
\hline
\begin{tabular}[c]{@{}l@{}}Item /\\  papel\end{tabular} & Desenvolvedor                                                        & Aplicações                                                                                                                                                                                                                                                                       \\\hline
1                                                       & \begin{tabular}[c]{@{}l@{}}Desenvolver\\  o código\end{tabular}      & \begin{tabular}[c]{@{}l@{}}A atividade de desenvolver incremento de software consiste \\ em escrever o código-fonte que implementa, \\ ou corrige, um ou mais requisitos do software. \\ Esta atividade pode ser \\ realizada mais de uma vez durante uma iteração.\end{tabular} \\\hline
2                                                       & \begin{tabular}[c]{@{}l@{}}Realizar teste \\ de Unidade\end{tabular} & \begin{tabular}[c]{@{}l@{}}Trata-se de um teste do tipo estrutural \\ (caixa-branca) que deve ser produzido pelo \\ programador, em parceria com o testador, antes da \\ implementação da unidade de software.\end{tabular}                                                      \\\hline
3                                                       & \begin{tabular}[c]{@{}l@{}}Corrigir \\ Bugs\end{tabular}             & \begin{tabular}[c]{@{}l@{}}Avaliar os resultados positivos \\ e negativos dos testes.\end{tabular}                                                                                                                                                                               \\\hline
4                                                       & \begin{tabular}[c]{@{}l@{}}Integração \\ do Software\end{tabular}    & \begin{tabular}[c]{@{}l@{}}desenvolver o sistema dividindo-o em módulos ou componentes, \\  funcionalidades do componente integrado devem \\ funcionar corretamente no sistema produzido.\end{tabular}                                                                           \\\hline
5                                                       & \begin{tabular}[c]{@{}l@{}}Teste de \\ Aceitação\end{tabular}        & \begin{tabular}[c]{@{}l@{}}Nesta etapa devem ser executados os testes de sistema e, \\ se aplicáveis, os testes de aceitação.\end{tabular}                                                                                                                                       \\\hline
6                                                       & \begin{tabular}[c]{@{}l@{}}Liberação \\ da Versão\end{tabular}       & \begin{tabular}[c]{@{}l@{}}Liberar uma versão testada \\ e atualizada do Incremento\end{tabular}                                                                           \\\hline                                                                                                     
\end{tabular}
\end{table}
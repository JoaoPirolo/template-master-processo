\begin{table}[h!]
\caption{Tabela tarefas do Gerente do Projeto}
\label{tab1}
\begin{tabular}{|l|l|l|}
\hline
\begin{tabular}[c]{@{}l@{}}Item /\\ papel\end{tabular} & Gerente de Projetos                                                                              & \multicolumn{1}{l|}{Aplicações}                                                                                                                                                                                                                                                                              \\\hline
1                                                      & \begin{tabular}[c]{@{}l@{}}Elaborar Termo de \\ Abertura do Projeto\end{tabular}                 & \begin{tabular}[c]{@{}l@{}}Formaliza o \\ inicio do Projeto\end{tabular}                                                                                                                                                                                                                                     \\\hline
2                                                      & \begin{tabular}[c]{@{}l@{}}Monitoramento e \\ Controle\end{tabular}                              & \begin{tabular}[c]{@{}l@{}}Assegurar que objetivos \\ sejam atingidos\end{tabular}                                                                                                                                                                                                                           \\\hline
3                                                      & \begin{tabular}[c]{@{}l@{}}-Controle de \\ Qualidade\end{tabular}                                & \begin{tabular}[c]{@{}l@{}}Definir padrões em procedimentos, \\ políticas e ações, de maneira uniforme.\end{tabular}                                                                                                                                                                                         \\\hline
4                                                      & \begin{tabular}[c]{@{}l@{}}-Validar o \\ Escopo\end{tabular}                                     & \begin{tabular}[c]{@{}l@{}}Processo de formalizar a \\ aceitação das entregas do projeto.\end{tabular}                                                                                                                                                                                                       \\\hline
5                                                      & \begin{tabular}[c]{@{}l@{}}-Controlar o\\  Escopo\end{tabular}                                   & \begin{tabular}[c]{@{}l@{}}Controlar as entregas do \\ Projeto\end{tabular}                                                                                                                                                                                                                                  \\\hline
6                                                      & \begin{tabular}[c]{@{}l@{}}-Controlar os \\ Riscos\end{tabular}                                  & \begin{tabular}[c]{@{}l@{}}Acompanhar os riscos identificados; Implementar os \\ planos de respostas aos riscos; Monitorar os \\ riscos residuais; Identificar novos riscos; \\ Avaliar a eficácia do processo de riscos durante o \\ ciclo de vida do projeto.\end{tabular}                                 \\\hline
7                                                      & \begin{tabular}[c]{@{}l@{}}-Controlar o \\ Cronograma\end{tabular}                               & \begin{tabular}[c]{@{}l@{}}Atualizar o progresso do projeto, \\ monitorar as variações entre o real com o planejado \\ (linha de base) e gerenciar as mudanças ocorridas.\end{tabular}                                                                                                                       \\\hline
8                                                      & \begin{tabular}[c]{@{}l@{}}-Controlar os \\ Custos\end{tabular}                                  & \begin{tabular}[c]{@{}l@{}}Monitorar o status do projeto para atualizar o orçamento e \\ gerenciar alterações na linha de base dos custos.\end{tabular}                                                                                                                                                      \\\hline
9                                                      & \begin{tabular}[c]{@{}l@{}}Identificar Partes \\ Interesadas\end{tabular}                        & \begin{tabular}[c]{@{}l@{}}Processo de identificação de todas as pessoas ou organizações que \\ podem ser afetadas pelo projeto e documentação das \\ informações relevantes relacionadas aos seus interesses, \\ envolvimento e impacto no sucesso do projeto.\end{tabular}                                 \\\hline
10                                                     & \begin{tabular}[c]{@{}l@{}}Atualização  da TAP \\ envio para o cliente\end{tabular}              & \begin{tabular}[c]{@{}l@{}}O termo de abertura do projeto deve conter \\ informações sumarizadas porém com o nível de \\ detalhamento necessário para a aprovação ou não do projeto.\end{tabular}                                                                                                            \\\hline
11                                                     & \begin{tabular}[c]{@{}l@{}}Encerramento\\  do Projeto\end{tabular}                               & \begin{tabular}[c]{@{}l@{}}Quando o projeto é encerrado o gerente de projeto deve finalizar \\ todos relatórios, documentar a experiência do projeto, fornecer \\ informação sobre o produto do projeto e como atendeu seus \\ requisitos do projeto, e então, documentar as lições aprendidas.\end{tabular} \\\hline
12                                                     & \begin{tabular}[c]{@{}l@{}}Elaborar Escopo \\ do Projeto\end{tabular}                            & \begin{tabular}[c]{@{}l@{}}Trabalho que precisa ser realizado para entregar um produto, \\ serviço ou resultado com as características e funções especificadas\end{tabular}                                                                                                                                  \\\hline
13                                                     & \begin{tabular}[c]{@{}l@{}}Elaboração do \\ Plano de \\ Gerenciamento \\ de Versões\end{tabular} & \begin{tabular}[c]{@{}l@{}}gerenciar diferentes versões no \\ desenvolvimento do software e do documento\end{tabular}                                                                                                                                                                                        \\\hline
14                                                     & \begin{tabular}[c]{@{}l@{}}Planejamento \\ da Versão\end{tabular}                                & \begin{tabular}[c]{@{}l@{}}planejar a versão do produto para a entrega, \\ estabelecendo um plano e a meta que o \\ Time e o resto da organização possam entender e comunicar.\end{tabular}                                                                                                                  \\\hline
15                                                     & \begin{tabular}[c]{@{}l@{}}Revisão \\ Planejamento \\ e Refinamento\end{tabular}                 & \begin{tabular}[c]{@{}l@{}}Essa tarefa compreende o planejamento da \\ homologação e a revisão geral do Plano de Projeto.\end{tabular}                                                                                                                                                                       \\\hline
16                                                     & \begin{tabular}[c]{@{}l@{}}Gerenciar \\ Comunicação\end{tabular}                                 & \begin{tabular}[c]{@{}l@{}}é o processo de colocar as informações \\ necessárias à disposição das partes \\ interessadas no projeto conforme planejado.\end{tabular}                                                                                                                                         \\\hline
17                                                     & Mobilizar Equipes                                                                                & \begin{tabular}[c]{@{}l@{}}tem como objetivo obter os recursos \\ humanos necessários para o projeto.
\end{tabular} \\\hline
\end{tabular}
\end{table}